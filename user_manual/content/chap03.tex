\chapter{Ansichten}

Neben den Verwaltungsansichten, die eingehend erwähnt wurden, existieren drei Ansichten zur Auswertung der Ein- und Ausnahmen:

\begin{itemize}[nosep]
	\item Monatsansicht (Startansicht)
	\item Jahresansicht
	\item Analyseansicht
\end{itemize}

\section{Monatsansicht}

Die Monatsansicht ermöglicht einen schnellen Überblick über alle Ausgaben eines Monats. Importierte Umsätze werden mit einem Icon gekennzeichnet, was somit anzeigt, dass eine Ausgabe bereits erfolgt ist. Das ist vor allem bei wiederkehrenden Ausgaben interessant, da dann der importierte Betrag in der Ansicht angezeigt wird -- gilt für alle Ansichten -- und nicht die geplante Ausgabe, wie sie im Tab "`Zahlungen"' in der Verwaltungsansicht der wiederkehrenden Ausgabe vorgemerkt ist (s. Abschnitt \ref{sec:fixedExpenses}).

Auf der rechten Seite werden alle aktiven Budgets mit deren Füllstand angezeigt.

\section{Jahresansicht}

Die Jahresansicht ermöglicht die Übersicht aller wiederkehrenden Umsätze über ein ausgewähltes Jahr in Form einer Matrix. Dabei sind zwei Betrachtungsweisen möglich:

\begin{itemize}[nosep]
	\item Die Summe der einzelnen Zeilen zeigt die (erwarteten) Ein- und Ausgaben je wiederkehrendem Umsatz an. 
	\item Die Summe der einzelnen Spalten zeigt den Saldo für den jeweiligen Monat an -- sofern Einnahmen im System gepflegt, ansonsten die Summe der Ausgaben.
\end{itemize}

Unten rechts in der Matrix findet sich das gesamte Saldo für das betrachtete Jahr. 

\textit{Anmerkung: Die Jahresansicht soll überarbeitet werden...}

\section{Analyseansicht}

Die Analyse Ansicht ermöglicht die Untersuchung der Ausgaben einer Umsatzkategorie, um die Entwicklung der Ausgaben innerhalb eines gewählten Zeitraums nachvollziehen zu können. Die Visualisierung erfolgt mithilfe eines Balkendiagramms, welches drei Achsen hat: Eine Abszissenachse (x-Achse) und zwei Ordinatenachsen (y-Achsen). 

Die x-Achse zeigt die Monate und die linke y-Achse zeigt die Höhe der Ausgaben in dieser Kategorie für den jeweiligen Monat. Die rechte x-Achse zeigt die kumulierte Höhe der Ausgaben für den dargestellten Zeitraum. Diese Kumulation wird als Linie auf dem Balkendiagramm dargestellt.   

Sollte ein Budget für die ausgewählte Kategorie gepflegt sein, wird dieses als horizontale, rot gestrichelte Linie angezeigt. Diese Linie richtet sich in Abhängigkeit des eingestellten Budgettyps ("`monatlich"' oder "`jährlich"') entweder an der linken oder rechten Abszissenachse aus. 

Ein Klick auf einen Balken befüllt die Tabelle auf der rechten Seite der Ansicht und deckt somit alle Ausgaben des Monats innerhalb dieser Kategorie auf.
